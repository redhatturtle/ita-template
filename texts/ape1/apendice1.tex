\chapter{Exemplo de Ap\^{e}ndice}

\section{Exemplo de Seção do Ap\^{e}ndice A}

Apêndice e anexos são opcionais no documento. O documento pode conter quantos apêndices ou anexos forem necessários. Lembrando que Apêndice é um documento ou texto elaborado pelo autor a fim de complementar sua argumentação e Anexo é um documento ou texto não elaborado pelo autor que servem de fundamentação ou comprovação (por exemplo: relatórios, mapas, leis, estatutos dentre outros). Os apêndices devem aparecer após as referências, e os anexos, após os apêndices, e ambos devem constar no sumário.
Caso tenha mais do que um apêndice e ou um anexo, deve-se utilizar a nomenclatura: Apêndice A, Apêndice B, Apêndice C etc.

\subsection{Exemplo de Subseção do Ap\^{e}ndice A}

\begin{figure}[h]
    \centering
    \includegraphics[height=5cm, width=5cm]{texts/ape1/figs/pragas_ciclo_cupim}
    \caption{Uma figura que está no apêndice}\label{FD}
\end{figure}
