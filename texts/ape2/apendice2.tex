\chapter{Exemplo de Segundo Ap\^{e}ndice}

\section{Exemplo de Seção do Segundo Ap\^{e}ndice A}

Apêndice e anexos são opcionais no documento. O documento pode conter quantos apêndices ou anexos forem necessários. Lembrando que Apêndice é um documento ou texto elaborado pelo autor a fim de complementar sua argumentação e Anexo é um documento ou texto não elaborado pelo autor que servem de fundamentação ou comprovação (por exemplo: relatórios, mapas, leis, estatutos dentre outros). Os apêndices devem aparecer após as referências, e os anexos, após os apêndices, e ambos devem constar no sumário.
Caso tenha mais do que um apêndice e ou um anexo, deve-se utilizar a nomenclatura: Apêndice A, Apêndice B, Apêndice C etc.

\subsection{Exemplo de Subseção do Segundo Ap\^{e}ndice A}

A matriz de Álgebra Linear $M$ e o vetor de torques inerciais $b$, utilizados na simulação são calculados segundo a formulação abaixo:
\begin{equation}
    M=\left[
        \begin{array}{ccc}
            M_{11} & M_{12} & M_{13} \\
            M_{21} & M_{22} & M_{23} \\
            M_{31} & M_{32} & M_{33}
        \end{array} \right]
\end{equation}