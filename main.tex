%%% Exemplo de utilização da classe ITA
%%%
%%%   por        Fábio Fagundes Silveira   - ffs            [at] ita   [dot] br
%%%              Benedito C. O. Maciel     - bcmaciel       [at] ita   [dot] br
%%%              Giovani Volnei Meinertz   - giovani        [at] ita   [dot] br
%%%              Hudson Alberto Bode       - bode           [at] ita   [dot] br
%%%              P. I. Braga de Queiroz    - pi             [at] ita   [dot] br
%%%              Jorge A. B. Gripp         - gripp          [at] ita   [dot] br
%%%              Juliano Monte-Mor         - jamontemor     [at] yahoo [dot] com [dot] br
%%%              Tarcisio A. B. Gripp      - tarcisio.gripp [at] gmail [dot] com
%%%              Alejandro Rios            - aarc.88        [at] gmail [dot] com
%%%              Saulo Gómez               - sagomezs       [at] unal  [dot] edu [dot] co
%%%              Ocimar Santos             - ocimar.acad    [at] gmail [dot] com
%%%              Fábio Malacco Moreira     - fabiom91       [at] gmail [dot] com
%%%
%%%   Versão para overleaf:
%%%   por        Alejandro A. Rios Cruz    - aarc.88@gmail.com
%%%              Saulo Gómez               - sagomezs@unal.edu.co
%%%              Ocimar Santos             - ocimar.acad@gmail.com
%%%
%%% Tese.tex  2024-03-11
%%% $HeadURL: https://github.com/redhatturtle/ita-template $
%%% $HeadURL: https://github.com/AlejandroRios/Template_Thesis_ITA $
%%% $HeadURL: http://www.apgita.org.br/apgita/teses-e-latex.php $
%%% $HeadURL: file:///opt/repositorioITALUS/classeITA/tags/versao-2.1/ExemploTeseITA.tex $
%%%
%%% ITALUS
%%% Instituto Tecnológico de Aeronáutica --- ITA, Sao Jose dos Campos, Brasil
%%% HomePages:        http://www.comp.ita.br/italus
%%%                   http://groups.yahoo.com/group/italus/
%%% Discussion list: italus {at} yahoogroups.com
%%%

%===============================================================================
% Para alterar o TIPO DE DOCUMENTO, preencher a linha abaixo \documentclass[?]{?}
%   \documentclass[dsc]{ita} = Tese de Doutorado
%                  quali     = Exame de Qualificacao
%                  msc       = Dissertacao de Mestrado
%                  tg        = Trabalho de Graduacao
%
% Para trabalhos em Inglês, adicionar 'eng':
%   \documentclass[dsc, eng]{ita}
%
% Para 'Draft Version'/'Versão Preliminar' com data no rodapé, adicionar 'dv':
%   \documentclass[dsc, eng, dv]{ita}
%===============================================================================
% As vezes após alterar as opções da classe é preciso compilar mais de 1 vez
\documentclass[dsc,eng]{ita} % ITA.cls based on standard book.cls

%===============================================================================
% Pacotes extras recomendados para teses
%===============================================================================

% AMS Packages
%\usepackage{amsmath}     % Ambiente de matemática e equações avançado
%\usepackage{amsfonts}    % Fontes p/ conjuntos: R, Z, H, etc
%\usepackage{amssymb}     % Símbolos matemáticos
%\usepackage{amsthm}      % Theorem writing tools

%\usepackage{epsfig}
%\usepackage{subfig}
%\usepackage{multirow}
%\usepackage{float}

%===============================================================================
% Identificações (se o trabalho for em inglês, insira os dados em inglês)
% Para entradas abreviadas de Professora (Profa.) em português escreva: Prof$^\textnormal{a}$.
%===============================================================================
% Cursos de Graduação do ITA:
% - Engenharia Aeronáutica
% - Engenharia Eletrônica
% - Engenharia Mecânica-Aeronáutica
% - Engenharia Civil-Aeronáutica
% - Engenharia de Computação
% - Engenharia Aeroespacial
%
% Programas de Formação Complementar (PFC):
% - PFC/F: Engenharia Física
% - PFC/I: Inovação
% - PFC/B: Bioengenharia
% - PFC/C: Engenharia de Controle e Automação
% - PFC/D: Ciência de Dados
%
% Programa de Pós-Graduação e áreas de concentração do ITA:
% - PG/EAM: Pós-Graduação em Engenharia Aeronáutica e Mecânica
%   - EAM-1: Projeto Aeronáutico, Estruturas e Sistemas Aeroespaciais
%   - EAM-2: Propulsão Aeroespacial e Energia
%   - EAM-3: Materiais, Manufatura e Automação
% - PG/EEC: Pós-Graduação em Engenharia Eletrônica e Computação
%   - EEC-D: Dispositivos e Sistemas Eletrônicos
%   - EEC-I: Informática
%   - EEC-M: Micro-ondas e Optoeletrônica
%   - EEC-S: Sistemas e Controle
%   - EEC-T: Telecomunicações
% - PG/FIS: Pós-Graduação em Física
%   - FIS-1: Física Nuclear
%   - FIS-2: Física Atômica e Molecular
%   - FIS-3: Física de Plasmas
%   - FIS-4: Dinâmica Não Linear e Sistemas Complexos
% - PG/EIA: Pós-Graduação em Engenharia de Infraestrutura Aeronáutica
%   - EIA-1: Infraestrutura Aeroportuária
%   - EIA-2: Transporte Aéreo e Aeroportos
% - PG/CTE: Pós-Graduação em Ciências e Tecnologias Espaciais
%   - CTE-F: Física e Matemática Aplicadas
%   - CTE-Q: Química dos materiais
%   - CTE-P: Propulsão Espacial e Hipersônica
%   - CTE-S: Sensores e Atuadores Espaciais
%   - CTE-E: Sistemas Espaciais, Ensaios e Lançamentos
%   - CTE-G: Gestão Tecnológica
% - PG/PO : Pós-Graduação em Pesquisa Operacional
\course{Engenharia Eletrônica e Computação}
\area{Informática}

% Divisão Acadêmica do ITA (não utilizado no template atual):
% - IEF: Divisão de Ciências Fundamentais
% - IEA: Divisão de Engenharia Aeronáutica
% - IEE: Divisão de Engenharia Eletrônica
% - IEM: Divisão de Engenharia Mecânica
% - IEI: Divisão de Engenharia Civil
% - IEC: Divisão de Ciência da Computação
%\dept{Engenharia Mecânica}

% Autor
% Gênero do autor(a): "mas" ou "fem"
\authorgender{fem}
% {FirstName}{LastName}
\author{Clarisse Sieckenius}{de~Souza}

% Endereço do Autor segundo guia de endereçamento dos correios
% #1 - Tipo + nome do logradouro + número. Ex: "Av. Cidade Jardim 679"
% #2 - CEP no formato 12345-123
% #3 - Cidade e sigla do estado, separados por hífen. Ex: "Campinas - SP"
\itaauthoraddress{Praça Mal Eduardo Gomes 50}
                 {12228-904}
                 {S\~{a}o Jos\'{e} dos Campos - SP}

% Titulo do Trabalho/Dissertação/Tese
\title{Modelo Latex para TG / Dissertação / Tese do ITA}

% Orientador
% Gênero do orientador(a): "mas" ou "fem"
\advisorgender{mas}
% [cargo]{título}{nome}{instituição}
\advisor{Prof.\ Dr.}{Adalberto Santos Dumont}{ITA}

% Coorientador
% Caso não haja coorientador, comentar/remover os commandos \coadvisorgender e \coadvisor
% Gênero do coorientador(a), "mas" ou "fem"
\coadvisorgender{fem}
% [cargo]{título}{nome}{instituição}
\coadvisor{Profa.\ Dra.}{Claudia M. Bauzer Medeiros}{Unicamp}

% Pró-reitor da Pós-graduação
% Gênero do Pró-Reitor(a): "mas" ou "fem"
\bossgender{mas}
% [cargo]{título}{nome}
\boss{Prof.\ Dr.}{John von Neumann}

%Coordenador do curso no caso de TG
% Gênero do Coordenador(a) do curso: "mas" ou "fem"
\bosscoursegender{mas}
% [cargo]{título}{nome}
\bosscourse{Prof.\ Dr.}{John Walker}

% Palavras-Chaves informadas pela Biblioteca -> utilizada na CIP
\kwcip{PalavraChave1}
\kwcip{PalavraChave2}
\kwcip{PalavraChave3}

% Membros da banca examinadora
% {título}{nome}{cargo na banca}{instituição}
\examiner{Prof.\  Dr. }{Richard Harbert Smith}{Presidente    }{ITA                    }
\examiner{Prof.\  Dr. }{Alan Turing          }{Membro Externo}{Princeton University   }
\examiner{Profa.\ Dra.}{Amalie Emmy Noether  }{Membro Externo}{University of Göttingen}
\examiner{        Dr. }{Linus Torwalds       }{Membro Interno}{Linux Foundation       }
\examiner{Prof.\  Dr. }{Jean Paul Jacob      }{Membro Interno}{ITA                    }

% Data da defesa (mês em maiúsculo, se trabalho em inglês, e minúsculo se trabalho em português)
\date{14}{março}{2024}

% Número CDU - (usando somente para TG)
%\cdu{621.38}

\begin{document}
  % Folha de Rosto e Capa para o caso do TG
  \maketitle

  % Dedicatória
  \begin{itadedication}
    Aos amigos da Graduação e Pós-Graduação do ITA por motivarem tanto a criação
    deste template pelo Fábio Fagundes Silveira quanto por motivarem a mim e
    outras pessoas a atualizarem e aprimorarem este excelente trabalho.
  \end{itadedication}

  % Agradecimentos
  \begin{itathanks}
    Primeiramente, gostaria de agradecer ao Dr. Donald E. Knuth e ao Dr. Leslie Lamport, por ter desenvolvido o \TeX e o \LaTeX.

Ao Prof. Dr. Orientador, pela orientação e confiança depositada na realização deste trabalho.

Ao Dr. Nelson D'Ávilla, por emprestar seu nome a essa importante via de trânsito na cidade de São José dos Campos.

  \end{itathanks}

  % Epígrafe
  \thispagestyle{empty}
  \ifhyperref\pdfbookmark[0]{\nameepigraphe}{epigrafe}\fi
  \begin{flushright}
    \begin{spacing}{1.15}
      \mbox{}\vfill
      {\sffamily\itshape
      ``If I have seen farther than others,\\
      it is because I stood on the shoulders of giants.''\\}
      --- \textsc{Sir Isaac Newton}
    \end{spacing}
  \end{flushright}

  % Resumo (em português)
  \begin{abstract}
    \noindent
    Aqui começa o resumo do referido trabalho. O resumo é a versão em Português do {\em abstract}, e como tal, segue as mesmas diretrizes de conteúdo.

  \end{abstract}

  % Abstract (em inglês)
  \begin{englishabstract}
    \noindent
    This is an example of an abstract.

An abstract summarizes the dissertation or thesis. Usually, It is one or two pages long. It describes the problem, motivates its solution and presents the proposed approach. An abstract must also describe the results obtained, and the whole text must be written in such a way that the reader can clearly identify what the thesis or dissertation is about, what is the proposed approach, and which results were obtained. The text style is formal and must avoid both self-contempt and self-acknowledgement. References should be avoided. The author must keep in mind that the abstract is probably the first part of the thesis or dissertation that any reader will read.

  \end{englishabstract}

  % Lista de figuras (automática e opcional)
  \listoffigures

  % Lista de tabelas (automática e opcional)
  \listoftables

  % Lista de abreviaturas (manual e opcional)
  \listofabbreviations
  \input{texts/cap0/abreviaturas}

  % Lista de símbolos (manual e opcional)
  \listofsymbols
  \begin{xltabular}{\textwidth}{l X}
    $a$              & Distância                                                               \\
    $\textbf{a}$     & Vetor de distâncias                                                     \\
    $\textbf{e}_{j}$ & Vetor unitário de dimensão $n$ e com o $j$-ésimo componente igual a $1$ \\
    $\textbf{K}$     & Matriz de rigidez                                                       \\
    $m_1$            & Massa do cumpim                                                         \\
    $\delta_{k-k_f}$ & Delta de Kronecker no instante $k_f$                                    \\
    $[x]$            & dimensão do vetor x                                                     \\
    $\sigma_{i}(A)$  & $i$-ésimo valor singular da matriz A                                    \\
    $A^{\#}$         & pseudo-inversa da matriz A                                              \\
    $(A)_{i}$        & $i$-ésima cluna da matriz A                                             \\
    $m_{i}$          & massa do $i$-ésimo link                                                 \\
    $I_{i}$          & inércia do $i$-ésimo link                                               \\
    $l_{i}$          & comprimento do $i$-ésimo link                                           \\
    $lc_{i}$         & distância entre a $i$-ésima junta e o centro de massa do $i$-ésimo link \\
    $J$              & matriz Jacobiana                                                        \\
    $M$              & matriz de inércia                                                       \\
    $W$              & inversa da matriz de inércia                                            \\
    $C$              & matriz de Coriolis e forças centrífugas                                 \\
    $G$              & vetor de forças gravitacionais                                          \\
    $\rho_{\tau}$    & índice de acoplamento de torque                                         \\
    $a$              & Distância                                                               \\
    $\textbf{a}$     & Vetor de distâncias                                                     \\
    $\textbf{e}_{j}$ & Vetor unitário de dimensão $n$ e com o $j$-ésimo componente igual a $1$ \\
    $\textbf{K}$     & Matriz de rigidez                                                       \\
    $m_1$            & Massa do cumpim                                                         \\
    $\delta_{k-k_f}$ & Delta de Kronecker no instante $k_f$                                    \\
    $[x]$            & dimensão do vetor x                                                     \\
    $\sigma_{i}(A)$  & $i$-ésimo valor singular da matriz A                                    \\
    $A^{\#}$         & pseudo-inversa da matriz A                                              \\
    $(A)_{i}$        & $i$-ésima cluna da matriz A                                             \\
    $m_{i}$          & massa do $i$-ésimo link                                                 \\
    $I_{i}$          & inércia do $i$-ésimo link                                               \\
    $l_{i}$          & comprimento do $i$-ésimo link                                           \\
    $lc_{i}$         & distância entre a $i$-ésima junta e o centro de massa do $i$-ésimo link \\
    $J$              & matriz Jacobiana                                                        \\
    $M$              & matriz de inércia                                                       \\
    $W$              & inversa da matriz de inércia                                            \\
    $C$              & matriz de Coriolis e forças centrífugas                                 \\
    $G$              & vetor de forças gravitacionais                                          \\
    $\rho_{\tau}$    & índice de acoplamento de torque                                         \\
    $a$              & Distância                                                               \\
    $\textbf{a}$     & Vetor de distâncias                                                     \\
    $\textbf{e}_{j}$ & Vetor unitário de dimensão $n$ e com o $j$-ésimo componente igual a $1$ \\
    $\textbf{K}$     & Matriz de rigidez                                                       \\
    $m_1$            & Massa do cumpim                                                         \\
    $\delta_{k-k_f}$ & Delta de Kronecker no instante $k_f$                                    \\
    $[x]$            & dimensão do vetor x                                                     \\
    $\sigma_{i}(A)$  & $i$-ésimo valor singular da matriz A                                    \\
    $A^{\#}$         & pseudo-inversa da matriz A                                              \\
    $(A)_{i}$        & $i$-ésima cluna da matriz A                                             \\
    $m_{i}$          & massa do $i$-ésimo link                                                 \\
    $I_{i}$          & inércia do $i$-ésimo link                                               \\
    $l_{i}$          & comprimento do $i$-ésimo link                                           \\
    $lc_{i}$         & distância entre a $i$-ésima junta e o centro de massa do $i$-ésimo link \\
    $J$              & matriz Jacobiana                                                        \\
    $M$              & matriz de inércia                                                       \\
    $W$              & inversa da matriz de inércia                                            \\
    $C$              & matriz de Coriolis e forças centrífugas                                 \\
    $G$              & vetor de forças gravitacionais                                          \\
    $\rho_{\tau}$    & índice de acoplamento de torque
\end{xltabular}


  % Índice, que o ITA chama de sumário (automático)
  \tableofcontents

  \mainmatter % Conteúdo principal
  \input{texts/cap1/capitulo1.tex} % Introdução
  \input{texts/cap2/capitulo2.tex} % Revisão Bibliográfica
  \input{texts/cap3/capitulo3.tex} % Metodologia
  \chapter{Resultados}

Caros amigos, a adoção de políticas descentralizadoras possibilita uma melhor visão global do levantamento das variáveis envolvidas. Nunca é demais lembrar o peso e o significado destes problemas, uma vez que a determinação clara de objetivos oferece uma interessante oportunidade para verificação das diretrizes de desenvolvimento para o futuro. Assim mesmo, a percepção das dificuldades cumpre um papel essencial na formulação do sistema de participação geral. Evidentemente, o desenvolvimento contínuo de distintas formas de atuação agrega valor ao estabelecimento das posturas dos órgãos dirigentes com relação às suas atribuições. Do mesmo modo, o novo modelo estrutural aqui preconizado auxilia a preparação e a composição do retorno esperado a longo prazo.

          Ainda assim, existem dúvidas a respeito de como a consolidação das estruturas deve passar por modificações independentemente de todos os recursos funcionais envolvidos. Podemos já vislumbrar o modo pelo qual a complexidade dos estudos efetuados facilita a criação do sistema de formação de quadros que corresponde às necessidades. Por outro lado, o consenso sobre a necessidade de qualificação prepara-nos para enfrentar situações atípicas decorrentes das novas proposições.

          Acima de tudo, é fundamental ressaltar que o início da atividade geral de formação de atitudes obstaculiza a apreciação da importância dos índices pretendidos. A prática cotidiana prova que o entendimento das metas propostas acarreta um processo de reformulação e modernização dos níveis de motivação departamental. Não obstante, o aumento do diálogo entre os diferentes setores produtivos garante a contribuição de um grupo importante na determinação das formas de ação. Todas estas questões, devidamente ponderadas, levantam dúvidas sobre se o comprometimento entre as equipes representa uma abertura para a melhoria da gestão inovadora da qual fazemos parte.

          Pensando mais a longo prazo, a mobilidade dos capitais internacionais promove a alavancagem do processo de comunicação como um todo. Desta maneira, a hegemonia do ambiente político talvez venha a ressaltar a relatividade das diversas correntes de pensamento. O cuidado em identificar pontos críticos na expansão dos mercados mundiais exige a precisão e a definição dos procedimentos normalmente adotados. Gostaria de enfatizar que o desafiador cenário globalizado maximiza as possibilidades por conta do impacto na agilidade decisória.

          Todavia, a crescente influência da mídia pode nos levar a considerar a reestruturação do fluxo de informações. O empenho em analisar a necessidade de renovação processual desafia a capacidade de equalização dos modos de operação convencionais. Neste sentido, a competitividade nas transações comerciais causa impacto indireto na reavaliação de alternativas às soluções ortodoxas.
 % Resultados
  \input{texts/cap5/capitulo5.tex} % Conclusão

  % Referências Bibliográficas
  \renewcommand\bibname{\itareferencesnamebabel} % Renomear título do capítulo referências
  \bibliography{bib/referencias.bib}

  \appendix % Apêndices (opcional)
  % Caso os apêndices não sejam utilizados comentar ou remover os comandos `input`
  \chapter{Exemplo de Ap\^{e}ndice}

\section{Exemplo de Seção do Ap\^{e}ndice A}

Apêndice e anexos são opcionais no documento. O documento pode conter quantos apêndices ou anexos forem necessários. Lembrando que Apêndice é um documento ou texto elaborado pelo autor a fim de complementar sua argumentação e Anexo é um documento ou texto não elaborado pelo autor que servem de fundamentação ou comprovação (por exemplo: relatórios, mapas, leis, estatutos dentre outros). Os apêndices devem aparecer após as referências, e os anexos, após os apêndices, e ambos devem constar no sumário.
Caso tenha mais do que um apêndice e ou um anexo, deve-se utilizar a nomenclatura: Apêndice A, Apêndice B, Apêndice C etc.

\subsection{Exemplo de Subseção do Ap\^{e}ndice A}

\begin{figure}[h]
    \centering
    \includegraphics[height=5cm, width=5cm]{texts/ape1/figs/pragas_ciclo_cupim}
    \caption{Uma figura que está no apêndice}\label{FD}
\end{figure}
 % Remover ou comentar
  \chapter{Exemplo de Segundo Ap\^{e}ndice}

\section{Exemplo de Seção do Segundo Ap\^{e}ndice A}

Apêndice e anexos são opcionais no documento. O documento pode conter quantos apêndices ou anexos forem necessários. Lembrando que Apêndice é um documento ou texto elaborado pelo autor a fim de complementar sua argumentação e Anexo é um documento ou texto não elaborado pelo autor que servem de fundamentação ou comprovação (por exemplo: relatórios, mapas, leis, estatutos dentre outros). Os apêndices devem aparecer após as referências, e os anexos, após os apêndices, e ambos devem constar no sumário.
Caso tenha mais do que um apêndice e ou um anexo, deve-se utilizar a nomenclatura: Apêndice A, Apêndice B, Apêndice C etc.

\subsection{Exemplo de Subseção do Segundo Ap\^{e}ndice A}

A matriz de Álgebra Linear $M$ e o vetor de torques inerciais $b$, utilizados na simulação são calculados segundo a formulação abaixo:
\begin{equation}
    M=\left[
        \begin{array}{ccc}
            M_{11} & M_{12} & M_{13} \\
            M_{21} & M_{22} & M_{23} \\
            M_{31} & M_{32} & M_{33}
        \end{array} \right]
\end{equation}

  \annex % Anexos (opcional)
  % Caso anexos não sejam utilizados comentar ou remover os comandos `input`
  \chapter{Exemplo de um Primeiro Anexo}

% Texto do Primeiro Anexo

\section{Uma Seção do Primeiro Anexo}
% Texto da primeira secao do primeiro anexo
Algum texto na primeira seção do primeiro anexo.


  \chapter{Exemplo de um Segundo Anexo}

% Texto do Segundo Anexo

\section{Uma Seção do Segundo Anexo}
% Texto da primeira secao do Segundo anexo
Algum texto na primeira seção do segundo anexo.



  %=============================================================================
  % Definição da Folha de Registro do Documento (FRD)
  %=============================================================================

  % Valores dos campos da FRD

  % Data do registro do documento
  \FRDitadata{26 de março de 2024}

  % Número de registro, fornecido pela biblioteca sob solicitação
  \FRDitadocnro{DCTA/ITA/TD-314/2024}
  \FRDitapalavrasautor{PalavraChave1; PalavraChave2; PalavraChave3}
  \FRDitapalavrasresult{PalavraChave1; PalavraChave2; PalavraChave3}
  % Difícil de usar variáveis pois essa pagina precisa ser sempre em português
  \FRDitapalavraapresentacao{
    ITA, \sjc.
    Curso de \itaoptioncourse.
    Programa de Pós-Graduação em \itaworkcourse.
    Área de \itaworkarea.
    Orientador(a): \itaadvisortitle\ \itaadvisorname.
    Coorientador(a):\ \itacoadvisorname.
    Defesa em 26/03/2024.
    Publicada em 26/03/2024.
  }
  \FRDitapalavraapresentacao{
    ITA, \sjc.
    Curso de XXX.
    Programa de Pós-Graduação em XXX.
    Área de XXX.
    Orientador: \itaadvisortitle\ \itaadvisorname.
    Coorientadora: \itacoadvisortitle\ \itacoadvisorname.
    Defesa em dd/mm/aaaa.
    Publicada em dd/mm/aaaa.
  }
  \FRDitaresumo{Aqui começa o resumo do referido trabalho. O resumo é a versão em Português do {\em abstract}, e como tal, segue as mesmas diretrizes de conteúdo.
}
  %  Primeiro Parâmetro: Nacional ou Internacional -- N/I
  %  Segundo parâmetro: Ostensivo, Reservado, Secreto ou Ultrassecreto -- O/R/S/U
  \FRDitaOpcoes{N}{O}
  % Cria o formulário
  \itaFRD
\end{document}
% Fim do Documento.
